% Options for packages loaded elsewhere
\PassOptionsToPackage{unicode}{hyperref}
\PassOptionsToPackage{hyphens}{url}
\PassOptionsToPackage{dvipsnames,svgnames,x11names}{xcolor}
%
\documentclass[
  letterpaper,
  DIV=11,
  numbers=noendperiod]{scrartcl}

\usepackage{amsmath,amssymb}
\usepackage{iftex}
\ifPDFTeX
  \usepackage[T1]{fontenc}
  \usepackage[utf8]{inputenc}
  \usepackage{textcomp} % provide euro and other symbols
\else % if luatex or xetex
  \usepackage{unicode-math}
  \defaultfontfeatures{Scale=MatchLowercase}
  \defaultfontfeatures[\rmfamily]{Ligatures=TeX,Scale=1}
\fi
\usepackage{lmodern}
\ifPDFTeX\else  
    % xetex/luatex font selection
\fi
% Use upquote if available, for straight quotes in verbatim environments
\IfFileExists{upquote.sty}{\usepackage{upquote}}{}
\IfFileExists{microtype.sty}{% use microtype if available
  \usepackage[]{microtype}
  \UseMicrotypeSet[protrusion]{basicmath} % disable protrusion for tt fonts
}{}
\makeatletter
\@ifundefined{KOMAClassName}{% if non-KOMA class
  \IfFileExists{parskip.sty}{%
    \usepackage{parskip}
  }{% else
    \setlength{\parindent}{0pt}
    \setlength{\parskip}{6pt plus 2pt minus 1pt}}
}{% if KOMA class
  \KOMAoptions{parskip=half}}
\makeatother
\usepackage{xcolor}
\setlength{\emergencystretch}{3em} % prevent overfull lines
\setcounter{secnumdepth}{-\maxdimen} % remove section numbering
% Make \paragraph and \subparagraph free-standing
\makeatletter
\ifx\paragraph\undefined\else
  \let\oldparagraph\paragraph
  \renewcommand{\paragraph}{
    \@ifstar
      \xxxParagraphStar
      \xxxParagraphNoStar
  }
  \newcommand{\xxxParagraphStar}[1]{\oldparagraph*{#1}\mbox{}}
  \newcommand{\xxxParagraphNoStar}[1]{\oldparagraph{#1}\mbox{}}
\fi
\ifx\subparagraph\undefined\else
  \let\oldsubparagraph\subparagraph
  \renewcommand{\subparagraph}{
    \@ifstar
      \xxxSubParagraphStar
      \xxxSubParagraphNoStar
  }
  \newcommand{\xxxSubParagraphStar}[1]{\oldsubparagraph*{#1}\mbox{}}
  \newcommand{\xxxSubParagraphNoStar}[1]{\oldsubparagraph{#1}\mbox{}}
\fi
\makeatother


\providecommand{\tightlist}{%
  \setlength{\itemsep}{0pt}\setlength{\parskip}{0pt}}\usepackage{longtable,booktabs,array}
\usepackage{calc} % for calculating minipage widths
% Correct order of tables after \paragraph or \subparagraph
\usepackage{etoolbox}
\makeatletter
\patchcmd\longtable{\par}{\if@noskipsec\mbox{}\fi\par}{}{}
\makeatother
% Allow footnotes in longtable head/foot
\IfFileExists{footnotehyper.sty}{\usepackage{footnotehyper}}{\usepackage{footnote}}
\makesavenoteenv{longtable}
\usepackage{graphicx}
\makeatletter
\def\maxwidth{\ifdim\Gin@nat@width>\linewidth\linewidth\else\Gin@nat@width\fi}
\def\maxheight{\ifdim\Gin@nat@height>\textheight\textheight\else\Gin@nat@height\fi}
\makeatother
% Scale images if necessary, so that they will not overflow the page
% margins by default, and it is still possible to overwrite the defaults
% using explicit options in \includegraphics[width, height, ...]{}
\setkeys{Gin}{width=\maxwidth,height=\maxheight,keepaspectratio}
% Set default figure placement to htbp
\makeatletter
\def\fps@figure{htbp}
\makeatother

\KOMAoption{captions}{tableheading}
\makeatletter
\@ifpackageloaded{caption}{}{\usepackage{caption}}
\AtBeginDocument{%
\ifdefined\contentsname
  \renewcommand*\contentsname{Table of contents}
\else
  \newcommand\contentsname{Table of contents}
\fi
\ifdefined\listfigurename
  \renewcommand*\listfigurename{List of Figures}
\else
  \newcommand\listfigurename{List of Figures}
\fi
\ifdefined\listtablename
  \renewcommand*\listtablename{List of Tables}
\else
  \newcommand\listtablename{List of Tables}
\fi
\ifdefined\figurename
  \renewcommand*\figurename{Figure}
\else
  \newcommand\figurename{Figure}
\fi
\ifdefined\tablename
  \renewcommand*\tablename{Table}
\else
  \newcommand\tablename{Table}
\fi
}
\@ifpackageloaded{float}{}{\usepackage{float}}
\floatstyle{ruled}
\@ifundefined{c@chapter}{\newfloat{codelisting}{h}{lop}}{\newfloat{codelisting}{h}{lop}[chapter]}
\floatname{codelisting}{Listing}
\newcommand*\listoflistings{\listof{codelisting}{List of Listings}}
\makeatother
\makeatletter
\makeatother
\makeatletter
\@ifpackageloaded{caption}{}{\usepackage{caption}}
\@ifpackageloaded{subcaption}{}{\usepackage{subcaption}}
\makeatother

\ifLuaTeX
  \usepackage{selnolig}  % disable illegal ligatures
\fi
\usepackage{bookmark}

\IfFileExists{xurl.sty}{\usepackage{xurl}}{} % add URL line breaks if available
\urlstyle{same} % disable monospaced font for URLs
\hypersetup{
  pdftitle={Najlepsze albumy muzyczne},
  colorlinks=true,
  linkcolor={blue},
  filecolor={Maroon},
  citecolor={Blue},
  urlcolor={Blue},
  pdfcreator={LaTeX via pandoc}}


\title{Najlepsze albumy muzyczne}
\usepackage{etoolbox}
\makeatletter
\providecommand{\subtitle}[1]{% add subtitle to \maketitle
  \apptocmd{\@title}{\par {\large #1 \par}}{}{}
}
\makeatother
\subtitle{wg użytkowników RateYourMusic.com}
\author{Bartosz Łuksza, Rafał Głodek}
\date{}

\begin{document}
\maketitle


\subsubsection{Wprowadzenie}\label{wprowadzenie}

Muzyka towarzyszy człowiekowi od tysięcy lat. Zawsze stanowiła
nieodłączną część naszej kultury. Niestety ograniczenia technologiczne
przez długi czas nie pozwalały artystom utrwalić swoich dzieł.
Fonografia narodziła się w XIX wieku, a jej największy rozkwit przypada
na drugą połowę wieku XX. Z tego względu najstarsze oryginalne dzieła
muzyczne, do których mamy obecnie dostęp, pochodzą poprzedniego
stulecia. W ostanich latach rynek muzyczny przeżywa niebywały rozwkit.
Każdego roku miliardy słuchaczy na całym świecie, przesłuchuje miliony
nowych albumów, generując przychody rzędu dziesiątek miliardów dolarów
ze sprzedaży nagrań. Rynek muzyczny jest jednak ściśle powiązany z
wieloma innymi gałęziami biznesu, takimi jak: film, moda, czy
technologie cyfrowe. Szacuje się, że każdego dnia na serwisy
streamingowe trafia nawet 120 000 utworów! W tej sytuacji można pokusić
się o stwierdzenie, że obecny przemysł muzyczny jest wręcz
``przeładowany'' muzyką. Warto zadać sobie pytanie, czy za ilością idzie
również jakość?

W naszej pracy zajrzymy wgłąb współczesnej historii muzyki i
przeanalizujemy bazę pięciu tysięcy najlepiej ocenianych albumów
muzycznych przez użytkowników RateYourMusic.com - największego portalu
do oceniania muzyki w internecie. Dane pochodzą z 12 grudnia 2021 r. i
zostały pobrane z serwisu kaggle.com. Wyodrębniliśmy z nich następujące
zmienne:

\begin{enumerate}
\item \textbf{\textit{Album}} - nazwa albumu
\begin{itemize}
\item Zawiera 4928 unikalne wartości
\end{itemize}
\item \textbf{\textit{Artist Name}} - artysta (imię i nazwisko lub pseudonim artystyczny)
\begin{itemize}
\item Zawiera 2787 unikalne wartości
\item 25 rekordów to "Various artists", czyli różni artyści, których jednak nie możemy wyodrębnić, więc pomijamy te wartości
\end{itemize}
\item \textbf{\texit{Release Date}} - dokładna data wydania albumu (dzień/miesiąc/rok), wyodrębniliśmy z niej dwie zmienne:
\begin{enumerate}
\item \textbf{\textit{Year}} - rok wydania albumu
\begin{itemize}
\item najmniejsza wartość: 1947
\item największa wartość: 2021
\item średnia arytmetyczna: 1987,46
\item mediana: 1988
\item wariancja: 253,23
\end{itemize}
\item \textbf{\textit{Month}} - miesiąc wydania albumu
\end{enumerate}
\item \textbf{\textit{Genres}} - gatunki (lub gatunek), do których należy album, np. Rock, Pop, Hip Hop
\begin{itemize}
\item Niekiedy trudno jest ustalić, do jakiego gatunku należy album. Wówczas określany jest mianem międzygatunkowego i klasyfikuje się go jako przynależnego do każdego z wymienionych gatunków.
\item Możliwe wartości to:
\end{itemize}
\end{enumerate}

Dogłębna analiza pozwoli nam znaleźć korelacje między różnymi zmiennymi
ujętymi w zestawieniu i wyciągnąć nieoczywiste wnioski. W ten sposób nie
tylko dowiemy się, jak muzyka rozwijała się w ubiegłych dekadach, ale
także nakreślimy scieżkę jej dalszego rozwoju.

W jakich latach powstawało najwięcej ``dobrych'' albumów? Czy istnieje
korelacja między średnią oceną użytkowników a datą wydania dzieła? Jakie
są średnie ocen dla różnych gatunków muzycznych? Którzy artyści mogą się
poszczycić najlepiej ocenianią dyskografią? Na te i wiele innych pytań
odpowiemy w naszej pracy.




\end{document}
